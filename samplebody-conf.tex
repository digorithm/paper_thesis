\section{Introduction}

The industrial adoption of microservices has led to increasingly complex configuration schemes that are commonly fine-tuned manually by engineers. Ganek and Corbi (2003) discussed the need for autonomic computing to handle the complexity of managing software systems, it had been noticed that managing complex systems has grown too costly, prone to error and labor-intensive, because people under such pressure make mistakes, increasing the potential of system outages with a concurrent impact on business [cite The dawning of the autonomic computing era].

Paul Horn in his IBM's autonomic computing initiative (2001) described software self-adaptation in terms of a few characteristics of an autonomic computing system, which he called the "grand challenge", here we reorganize these characteristics into the following terminology:

\begin{itemize}
  \item \textbf{Self-Awareness}: an autonomic system needs to know itself;
  \item \textbf{Self-configuring}: an autonomic system must configure and reconfigure itself under varying and unpredictable conditions;
  \item \textbf{Self-optimization}: an autonomic system never settles for the status quo, it always looks for ways to optimize its workings, it will anticipate the optimized resources needed to meet a user's information needs while keeping its complexity hidden;
  \item \textbf{Self-healing}: An autonomic system must perform something akin to healing, it must be able to recover from routine and extraordinary events that might cause some parts to malfunction.
\end{itemize}

Although this has driven many researchers to study self-adaptive systems over the years [cite them], the software industry still lacks practical tools to provide self-adaptation mechanisms to their systems, and thus, most of the configuring and tuning of the systems happens manually, often in run-time, which is known to be a very time consuming and risky practice.

In this work we focus on the practical realization of self-configuring and self-optimization, based on ideas already discussed, but following alternatives approaches to provide an accessible tool to build self-adaptive systems, such approaches are:

\begin{itemize}
  \item Use of system observability to collect a sufficiently big time series dataset that represents the state of a system and its components in relation to time;
  \item Use of Machine Learning applied to time series data to analyze and predict workload and optimal configuration in order to provide adaptation plans;
  \item Use of Online Learning to provide constant relearning of the model to adapt to different scenarios and requirements;
  \item Use of Service Level Objectives as a way to define the system's performance goals that will be used by the tool as optimization objectives;
  \item Use of the concepts in control theory as the central component of this tool, in which it implements a variation of the well known and studies MAPE loop (Monitor, Analyze, Plan, Execute);
  \item Abstraction of all the concepts above into a tool that can be integrated to compatible systems, including systems that were built without self-adaptation in mind. 
\end{itemize}

The rest of the paper is structure as following: 


% I can develop a bit more on this by using this survey's result: According to a recent
% IT resource survey by the Merit Project of Computer Associates International, 1867 respondents grouped the most common causes of outages into four areas of data center operations: systems, networks, database, and applications.
% Most frequently cited outages included:
%- For systems: operational error, user error, third-party software error, internally %developed software problem, inadequate change control, lack of automated processes
%- For networks: performance overload, peak load problems, insufficient bandwidth
%- For database: out of disk space, log file full, performance overload
%- For  applications:  application  error,  inadequate change control, operational %error, nonautomated application exceptions
% A nice point is that these are the problems we're still facing whenever we deal poorly with the complexity of the systems

% It would be nice also to cite IBM autonomic computing initiative

% IBM’s autonomic computing initiative has been outlined broadly. Paul Horn described this “grand challenge” and called for industry-wide collaboration toward developing autonomic computing systems that have characteristics as follows: ● To be autonomic  a system needs to “ know itself ”— and consist of components that also possess a sys- tem identity. ● An autonomic system must con fi gure and recon- fi gure itself under varying and unpredictable con- ditions. ● An autonomic system never settles for the status quo — it always looks for ways to optimize its work- ings. ● An autonomic system must perform something akin to healing — it must be able to recover from routine and extraordinary events that might cause some parts to malfunction. ● A virtual world is no less dangerous than the phys- ical one, so an autonomic computing system must be an expert in self-protection. ● An autonomic computing system knows its envi- ronment and the context surrounding its activity, and acts accordingly. ● An autonomic system cannot exist in a hermetic environment (and must adhere to open standards). ● Perhaps most critical for the user, an autonomic computing system will anticipate the optimized re- sources needed to meet a user ’ s information needs while keeping its complexity hidden.

\section{Related Work}

Maybe related work would be better now instead of putting it in the end, since there are a lot of related work and they are very relevant to this work, it would be nice to upfront point out what has been done and what would be different in this work.


\section{Approach}

\subsection{Control theory and self-adaptive systems}

\subsection{System's configuration as an optimization problem}

\subsection{Providing system adaptation with machine learning}

\subsection{Workload simulation}

\subsection{System instrumentation}

\subsection{Machine learning architecture}

\subsubsection{Features and models}

\subsubsection{Online training}

\subsubsection{Achieving self-adapation}

\section{Implementation}

\section{Evaluation and discussion}

\section{Future work}

\section{Conclusions}

%\end{document}  % This is where a 'short' article might terminate

\nocite{*}

\begin{acks}

\end{acks} 