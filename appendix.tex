\begin{appendices}
\chapter{More on why using Go and Python}
Companies usually try to enforce homogeneous codebases, in other words: using a single language or the minimum possible number of languages, to both make it easier to maintain the codebase and to make it easier to hire engineers.

However, in this work we have decided to use 2 languages, the reason for that is that we believe we should use the right tools for a given job.

Machine learning tooling in Python is complete and backed by a strong community, the fact that Python is a dynamically typed languages makes total difference when performing data science experiments; we usually need to add and remove features as quickly as possible, and this can become very time demanding in a statically typed language with strict policies.

On the other hand, distributed and concurrent programming in Golang is as natural as the language itself can be; after all, these programming constructs are embedded in the language. Enforcing reliability in a system built in Golang is significantly easier than in systems built in Python.

And since \projectname{} needs to provide reliability, speed, and a fast way to experiment with machine learning using an arbitrary dataset, using Python for machine learning and wrapping its service and provide a solid infrastructure using Golang was the way to go.

The communication between these two components could be done using many common strategies such as exposing the machine learning service as a REST API, or using gRPC. However, that would introduce unnecessary complexities such as an HTTP and/or TCP layer, where all the communication between them is going to happen in the same machine---at least for now. Thus, exposing the machine learning component through a clean CLI-like API, and controlling it in Golang using Bash is the easiest and fastest way to achieve this communication.

% TODO: create image graph to illustrate this
\end{appendices}